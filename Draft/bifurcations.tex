% ****** Start of file apssamp.tex ******
%
%   This file is part of the APS files in the REVTeX 4.2 distribution.
%   Version 4.2a of REVTeX, December 2014
%
%   Copyright (c) 2014 The American Physical Society.
%
%   See the REVTeX 4 README file for restrictions and more information.
%
% TeX'ing this file requires that you have AMS-LaTeX 2.0 installed
% as well as the rest of the prerequisites for REVTeX 4.2
%
% See the REVTeX 4 README file
% It also requires running BibTeX. The commands are as follows:
%
%  1)  latex bifurcations.tex
%  2)  bibtex bifurcations
%  3)  latex bifurcations.tex
%  4)  latex bifurcations.tex
%
\documentclass[%
 % reprint,
10pt,
superscriptaddress,
twocolumn,
%groupedaddress,
%unsortedaddress,
%runinaddress,
%frontmatterverbose,
%preprint,
%preprintnumbers,
%nofootinbib,
%nobibnotes,
%bibnotes,
 amsmath,amssymb,
 aps,prx,
%pra,
%prb,
%rmp,
%prstab,
%prstper,
%floatfix,
]{revtex4-2}

\usepackage{graphicx}% Include figure files
\usepackage{dcolumn}% Align table columns on decimal point
% \usepackage{bm}% bold math
\usepackage{xcolor}
\newcommand{\red}[1]{\textcolor{red}{#1}}
\newcommand{\blue}[1]{\textcolor{blue}{#1}}
\usepackage{hyperref}% add hypertext capabilities
\usepackage{siunitx}% SI units
\usepackage{physics}
\usepackage[utf8]{inputenc}
\usepackage[T1]{fontenc}
\usepackage{lmodern}
\usepackage{amsmath,amsfonts,amssymb}
\usepackage{natbib}


\AtBeginDocument{\renewcommand*{\d}{\mathop{\kern0pt\mathrm{d}}\!{}}}
\graphicspath{{Figures/}} % set default figure path to figures/, if we store figure files in figures/, we only need to put file name in \includegraphics{filename.pdf}. For supported formats, e.g. pdf and jpg, we can omit the extensions. This enhances the readability of the source.

% Some formatting guidelines
% 1. Start a new line (\n) for each sentence. This is good for synctex (click pdf and find the line in tex), and also good for Git when comparing versions.
% 2. Communicate thoughts in comments. For example, if you think a figure of something is needed but missing some where, put a comment and describe the needs.
% 3. Colored text: I use red text to emphasize that the claim is not fully backed by our results. The wording may be modified in the future.
% 4. Figure crossref: at the beginning of a sentence, use "Figure~\ref{}". Otherwise, use "Fig.~\ref{}". Note that the "~" is to prevent line breaking in the middle of the crossref.

\begin{document}

\preprint{APS/123-QED}

\title{Active Nematics at Bifurcations}% Force line breaks with \\

\author{Zhengyang Liu}
\author{Claire Doré}
\affiliation{Laboratoire Gulliver, UMR 7083 CNRS, ESPCI Paris, PSL Research University, 75005 Paris, France.}

\author{Antonio Tavera-Vazquez}
\affiliation{Laboratoire Gulliver, UMR 7083 CNRS, ESPCI Paris, PSL Research University, 75005 Paris, France.}
\affiliation{Pritzker School of Molecular Engineering, University of Chicago, Chicago, IL 60637, USA.}

\author{Teresa Lopez-Leon}
\affiliation{Laboratoire Gulliver, UMR 7083 CNRS, ESPCI Paris, PSL Research University, 75005 Paris, France.}
\date{\today}


\begin{abstract}

Under lateral confinement, active matter self-organize into coherent flows. 
Such behavior implies the possibility of achieving logical operations in properly designed channel networks. 
Bifurcations are a key ingredient in channel networks.
Understanding active matter behavior at bifurcations is therefore an important step towards a proper channel network design.
In this paper, we experimentally explore active matter behavior at bifurcations using the microtubule-kinesin model system. 
Specifically, we compare the effects of channel length, ratchets and turning angles. 
Our results suggest that ratchets and turning angles help establish unambiguous polarized flow states.
In contrast, channel length is a less relevant factor, which results in more frequently changing flow states.
Our experiment is the first step to understanding active nematic flows in complex channel networks.
The result lays the foundation for active matter logic and computation.

\end{abstract}

\keywords{active matter logic, confinement, active nematics, ratchet, bifurcation}

\maketitle

\section{Introduction}

Active matter flows spontaneously under channel confinement, forming coherent flows  \cite{Wioland2016,Wu2017,Morin2018,Hardouin2019,Hardouin2020}. 
Such behavior implies several possibile applications of active matter, including serving as micro-scale transport, soft robotics and active matter logic \cite{Thampi2022,Woodhouse2017}.
Boundary-mediated control has been shown effective in manipulating active matter in both experiments \cite{Lushi2014,Wioland2016,Wioland2016,Wu2017,Morin2018,Liu2019,Ross2019,Hardouin2019} and simulations \cite{Voituriez2005,Marenduzzo2007,Shendruk2017,Vaidya2024}.
As of now, most studies have focused on the behavior of active matter in stand-alone smooth channels, which showed that active flows were intrinsically bistable \cite{Wu2017,Morin2018}.
However, to realize the full potential of active matter channel flows, it is necessary to study the behavior of active matter in channel networks and with asymmertic geometries, as suggested by the pioneering theoretical work on active matter logic \cite{Woodhouse2017}.
Very recently, channel networks attract more attention, and frustrated flow states have been investigated in coupled annular rings \cite{Hardouin2020} and large honeycomb-like networks \cite{Jorge2024}.
The other essential component of active matter logic is the diode channel, which only permits flow in one direction.
While a few early works have hinted or employed asymmetric geometries, such as a kink or an array of ratchet teeth, to steer active matter flows \cite{ElizabethHulme2008,DiLeonardo2010,Wu2017,Hardouin2020,Ray2023,Vaidya2024}, a systematic study of asymmetric channels in the context of channel networks, especially frustrated flow states, is still missing. 

% Spontaneous flow under channel confinement is a phenomenon observed in various active matter systems \cite{Lushi2014,Wioland2016,Wu2017,Duclos2017,Morin2018,Hardouin2020}.
% Due to its ubiquitous and robustness, active matter holds significant potential for many applications, especially in the fields of microfluidics and soft robotics \cite{Hardouin2020}.
% Among the various proposed applications, active flow networks (AFN) are particularly interesting \cite{Woodhouse2016,Woodhouse2017}.
% AFNs are networks of channels where active matter flows.
% The most notable feature of AFNs is the potential capability of achieving logical operations, such as AND and OR, which enables the combination of mass transport and intelligence in a single circuit system. 
% \citet{Woodhouse2017} proposed to use a Landau-type bistable potential to model the phenomenon. 
% Combining mass conservation and diode channel which only permits flow in certain direction, they derived a theoretical framework of AFNs that can achieve logical operations.

% \begin{figure}[!h]
%     \includegraphics[width=0.45\textwidth]{1-bifurcation-question}
%     \caption{
%     \textbf{What do active nematics do in straight channels and bifurcations?}
%     (a) Spontaneous directed flow in a straight channel.
%     (b) A schematic diagram of the directed flow in a straight channel.
%     (c) Energy landscape of flow rate in a straight channel predicted by a Landau-type phenomenological model, relating flow potential $V(\phi)$ and flow rate $\phi$ of active flows. 
%     The red disk represents the most probable flow in the positive direction, corresponding to the scenario in a and b.
%     (d) A schematic diagram of three interconnected straight channels, the so called ``bifurcation''.
%     (e) Energy landscape of the flow rate configurations in the bifurcation. 
%     The blue dot and the lower left inset illustrate a typical ``polarized'' flow state, where the in-coming flow from one channel completely goes into one of the two outlet channels without splitting. 
%     The red dot and the upper right inset illustrate a typical ``non-polarized'' flow state, where the in-coming flow from one channel equally splits the two outlet channels. 
%     }
%     \label{fig:bifurcation-question}
% \end{figure}

% A key feature of AFNs is the bifurcation, where a channel splits into two or more channels. 
% According to the model by \citet{Woodhouse2017} and the experiment by \citet{Morin2018}, active matter flow in channels have a preferred flow rate $\phi_0$, which depends only on channel width and activity, as illustrated in Figs.~\ref{fig:bifurcation-question} (a) and (b).
% Such a preferred flow rate can be modeled by a Landau-type bistable potential, as shown in Fig.~\ref{fig:bifurcation-question} (c).
% When three channels of identical width are connected to one node, i.e. at a bifurcation (as in Fig.~\ref{fig:bifurcation-question}(d)), however, the all-$\phi_0$ flow state is frustrated by mass conservation.
% In such a frustrated state, the model by \citet{Woodhouse2017} predicts that the most favorable flow state is that the flow enters the node at flow rate $\phi_0$ from one channel and exits at the same flow rate $\phi_0$ through another channel, leaving the flow rate in the third channel $0$.
% Figure~\ref{fig:bifurcation-question}(e) shows the predicted flow state histogram.
% Each point in this histogram represents a flow configuration represented by flow rates in channel B $\phi_B$ and channel C $\phi_C$, and the flow rate in channels A satisfies $\phi_A+\phi_B+\phi_C=0$.
% The blue dot represents one of the six most probable flow state, where the flow enters the node from channel A and exits through channel B, without splitting into channel C.
% Furthermore, if the two outlet channels are of different lengths, the exit flow follows the longer path.  
% This behavior is the fundation to achieve logical operation with active matter confined in channel networks. 

% Despite the theoretical progress and the interesting promise of AFNs, experimental realization of AFNs is very rare due to the technical challenges in fabricating and properly applying the confinement structure to active matter.

% First, this realization provides a playground to test and improve existing theories, and thus deepen our understanding of active matter behavior in complex environment. 
% Second, this realization lays the foundation for potential applications of active flow networks in mass transport and flow computation. 

In this work, we filled this gap by experimentally studying the flow behavior of active matter at channel networks consisting of asymmetric channels. 
To obtain a clear understanding, we studied the simplest possible form of a channel network -- the bifurcation -- where three channels are connected to one node.
Despite of being simple, the bifurcation is a key element of more complex channel networks, and a great system to observe frustrated flow states.
\red{Assymetry are in two levels: in single channel level, ratchets are introduced to favor the flow in a certain direction; in network level, input channels are designed to split into channels with different lengths, number of ratchets and turning angles.}
Our results suggest that ratchets and turning angles help establish stable polarized flow states, where the inlet flow primarily goes into one of the outlet channels, leaving the other channel with little flow.
In contrast, channel length is a less relevant factor, which results in more frequently changing flow states.
\red{The topological defects show different dynamics in flowing ratchet channels from frustrated straight channels, uncovering the steering mechanism of the ratchets.}
Our experiment is the first step to understanding active nematic flows in complex channel networks.
The result lays the foundation for potential applications of active flow networks in mass transport and flow computation. 

\section{Experiment}

\begin{figure}[!h]
    \includegraphics[width=0.45\textwidth]{2-bifurcation-experiment}
    \caption{
    \textbf{Confining microtubule-kinesin system at water-oil inerface -- the experimental setup.}
    (a) Schematic diagram of the experimental setup. The microtubule-kinesin active nematic system is placed at an water-oil interface in a custom PDMS pool, and is subject to lateral confinement by the micro-printed bifurcation channel. 
    (b) Confocal image of a mature interfacial microtubule-kinesin system. 
    (c) Confocal image of the bifurcation channels set on the interfacial microtubule-kinesin system. 
    }
    \label{fig:bifurcation-experiment}
\end{figure}

Our active matter system comprises microtubule filaments powered by ATP-consuming two-headed kinesin molecular motors \cite{Sanchez2012}. 
By adding depleting agent poly ethylene glycol (PEG), the system forms dense bundles at the water-oil interface.
Driven by the kinesin motors, the bundles stretch and bend constantly, exhibiting chaotic flows characterized by the formation and annihilation of topological defects.
In an experiment, \SI{2.5}{\micro\liter} of microtubule solution was put in a custom pool of \SI{5}{\milli\meter} diameter, covered by \SI{100}{\micro\liter} of silicone oil (5 cSt).
The micro-printed bifurcation channel structure was then gently placed at the oil-water interface, confining the chaotic system into channel flows (Fig.~\ref{fig:bifurcation-experiment}(a)). 
An unconfined active nematics system is shown in Fig.~\ref{fig:bifurcation-experiment}(b), while the same system confined by the bifurcation channels is shown in Fig.~\ref{fig:bifurcation-experiment}(c).
The active nematics system is observed using a confocal microscope (Nikon), and images are taken at 2 Hz using a 10X objective lens.
Then, \SI{400}{\micro\meter} of each channel is cropped and analyzed by PIV, as shown in Fig.~\ref{fig:bifurcation-symmetric}(a).

\section{Results}

\begin{figure}[htb]
    \includegraphics[width=.45\textwidth]{3-bifurcation-symmetric}
    \caption{
    \textbf{Flow rate measurements and flow rate histogram.}
    (a) A snapshot of microtubule-kinesin system confined in bifurcation channels.
    The scale bar is 200 $\mu$m. The rectangles indicate the regions where PIV analysis was performed.
    (b) Zoom-in view of channel A, yellow arrows indicates local velocity from PIV analysis. 
    (c) Flow rate time series in the 3 channels A (blue), B (orange) and C (green). 
    The light curves in the back are the real flow rates, while the bold curves in the front are Gaussian-smoothed flow rates with $\sigma=50\;\mathrm{s}$. The ``normalizer'' and the sum of all flow rates are shown as red and gray, respectively. 
    (d-f) Raw, normalized and theoretical flow rate histograms of channels B (vertical) and C (horizontal). 
    }
    \label{fig:bifurcation-symmetric}
\end{figure}

\subsection{Symmetric bifurcation}

We first studied the flow behavior at a symmetric bifurcation, where all the channels are smooth and of the same length.
To extract the flow rate in each channel, we cropped regions \SI{400}{\micro\meter} from the connecting node for each channel and performed Particle Image Velocimetry (PIV) analysis (as indicated by the rectangles in Fig.~\ref{fig:bifurcation-symmetric}(a)).
Snapshots of the PIV results are shown in Fig.~\ref{fig:bifurcation-symmetric}(b).
Local velocity vectors $v(x,y)$ are indicated as yellow arrows, where $x$ and $y$ are defined separately as the transverse and parallel directions for each channel, respectively.

The surface flow rate in the channel direction ($Q_y$) was calculated by integrating the velocity field over the channel cross-section.
To minimize the noise in data, we also average the flow rate calculated at different $y$ positions.
Formally, our channel flow rate is defined as
%
\begin{equation}
    \phi = \left< Q_y \right> = \frac{1}{L}\int_0^{L} Q_y dy,
\end{equation}
%
where $Q_y=\int_{x_1}^{x_2} v_y(x, y)dx$ is the flow rate at $y$ position, $L$ is the length of the cropped image, and $x_1$ and $x_2$ are the left and right boundaries of the channel, respectively.
For consistency, we always define positive flow rate as the flow away from the connecting node.
With this definition, the mass conservation at the connecting node can be expressed as $\phi_A+\phi_B+\phi_C=0$.

The flow rates in channels A, B and C over time are plotted in Fig.~\ref{fig:bifurcation-symmetric}(c) in blue, orange and green curves, respectively, corresponding to the colors of the rectangles in Fig.~\ref{fig:bifurcation-symmetric}(a).
The light curves in the back are the real flow rates, while the bold curves in the front are Gaussian-smoothed flow rates with $\sigma=25$ s.
The gray curve is the sum of the flow rates in the 3 channels, $\phi_A+\phi_B+\phi_C$, which serves as a check for mass conservation.
The magnitude of the gray curve is much smaller than the flow rates in the 3 channels, indicating that mass conservation is indeed satified at the connecting node.
Figures~\ref{fig:bifurcation-symmetric}(d) show the histograms of raw flow rates in channels B and C.
Darker colors indicate higher probability of flow rate configurations, and the crossing points of the dashed lines are the origin of the histogram ($\phi_B=\phi_C=0$).
We only show the histogram of the flow rates in channels B and C, because the flow rates are constrained by $\phi_A+\phi_B+\phi_C=0$, meaning that two flow rates are enough to specify the flow configuration.
Since we were interested in the flow configurations, that is, how a flow in one channel was distributed in the other two channels, we focused on the normalized flow rates $\tilde\phi_A$, $\tilde\phi_B$ and $\tilde\phi_C$.
This eliminated the effect of the fluctuating overall flow rates (for detailed normalization procedure, see Supplemental Information).
The normalized flow rate histograms are shown in Fig.~\ref{fig:bifurcation-symmetric}(e).
Flow configurations that conserve mass is indicated by elliptic hexagons in dotted lines.
The normalized flow rate histogram show good agreement with the mass conservation line, which is reassuring for our experimental and analytical techniques.
It is possible to obtain the theoretical flow rate histogram by solving the Landau-type model proposed by \citet{Woodhouse2017}.
In short, the model states that the flow in a channel network tends to minimize the total energy of the system, which comprises channel flow energy, diode energy and mass conservation energy.
Formally, the Hamiltonian of the system to be minimized is
%
\begin{equation}
    H = H_{\mathrm{channel}} + H_{\mathrm{diode}} + H_{\mathrm{mass}},
\end{equation}
%
the detailed form of the Hamiltonian can be found in the original paper \cite{Woodhouse2017} and \red{Supplemental Information}.
By simulating the flow configurations in a Monte Carlo process, we obtained a theoretical flow rate histogram, as shown in Fig.~\ref{fig:bifurcation-symmetric}(f).
Compared to the theoretical flow rate histogram in Fig.~\ref{fig:bifurcation-symmetric}(f), the experimental histogram shows a broader distribution, covering most of the possible configurations of flow rates in the 3 channels.
Sharp peaks at polarized flow configurations, where one of the channels is a completely frustrated, however, were not observed in our experiment. 
It is worth noting that although the channels were designed to be fully symmetric, the grid requires a base structure to which the micromanipulator is attached (see Fig.~\ref{fig:bifurcation-experiment}(c)), which may result in asymmetry in the flow rates.
This is probably the reason why we did not observe all the flow configurations that satisfy mass conservation, especially those where the flow in channel A goes outwards ($\phi_A> 0$).

\subsection{Asymmetry in channel length}

\begin{figure}[t]
    \includegraphics[width=.45\textwidth]{bifurcation-asymmetric.pdf}
    \caption{
    \textbf{Straight channel length effect.}
    (a) A confocal fluorescence image of a bifurcation channel system with a 4-teeth ratchet inlet channel A and two straight outlet channels B and C with different lengths. The mean width of all channels is $150\;\mathrm{\mu m}$. 
    (b) A schematic diagram of the bifurcation channel system. The lengths of channel B is $1500\;\mathrm{\mu m}$ and the lengths of channels A and C are both $500\;\mathrm{\mu m}$.
    (c) The time series of flow rates in all channels. The normalizer and the sum of all flow rates are shown as red and gray, respectively. 
    (d) The flow rate histogram of channels B (vertical) and C (horizontal). The limits of the histogram are set to $[-2000, 2000]$ for both axes.
    (e) The normalized flow rate histogram. The limits of the histogram are set to $[-2, 2]$ for both axes. The hexagon in dotted lines indicates the configurations that conserve mass.
    (f) The theoretical flow rate histogram. 
    }
    \label{fig:bifurcation-asymmetric}
\end{figure}

Channel length plays a crucial role in determining the flow behavior in channel networks, according to the theoretical model by \citet{Woodhouse2017}.
Longer channels have a deeper energy well, which favors the flow to follow the longer path.
To test this hypothesis, we designed a bifurcation channel system with two outlet channels of different lengths, as shown in Fig.~\ref{fig:bifurcation-asymmetric}(a) and sketched in Fig.~\ref{fig:bifurcation-asymmetric}(b).
Channel A is modified with ratchets to ensure that channel A is always the inlet channel. 
Channel B is \SI{1500}{\micro\meter} long, while channels A and C are both \SI{500}{\micro\meter} long.
All the channels have the same mean width of \SI{150}{\micro\meter}.
The flow rates in the 3 channels are shown in Fig.~\ref{fig:bifurcation-asymmetric}(c).
The ratchet structure is effective in polarizing the flow, as the flow rate in channel A is always negative, meaning that the flow is always directed towards the center node. 
At the bifurcation, neither of the outlet channels is apparently preferred.
Instead, the flow splits almost equally between the two outlet channels.
The shorter channel, C, has a slightly higher flow rate than the longer channel, B.
This flow configuration is visualized in the raw and normalized flow rate histogram in Fig.~\ref{fig:bifurcation-asymmetric}(d) and (e), respectively.
The prediction from the theoretical model is shown in Fig.~\ref{fig:bifurcation-asymmetric}(f), where a sharp peak is observed at the polarized flow configuration, showing a sharp contrast to the experimental result. 
This result shows that the longer path is not necessarily preferred.

\subsection{Ratchet inlet and outlets}

\begin{figure*}[t]
    \includegraphics[width=\textwidth]{bifurcation-ratchet.pdf}
    \caption{
    \textbf{Ratchet inlet and outlets: histogram and time series.}
    (a) The numbers of ratchet teeth in channels A, B and C are 4, 14 and 4, respectively. 
    The splitting ratio is around 3:1.
    (b) 4-4-4 bifurcation, where channel B has an extended straight portion. 
    The flows again exhibit a sharp peak in the histogram at a splitting ratio around 1:1.
    (c) 5-5-5 bifurcation, where all the channels are of the same length. 
    The flows again exhibit a sharp peak in the histogram at a splitting ratio around 1:1.
    }
    \label{fig:bifurcation-ratchet}
\end{figure*}

The result from the asymmetric channel length experiment suggests that the ratchet structures play a more dominant role than channel length.
We therefore use ratchet structures to modify both the inlet and outlet channels.
The numbers of ratchet teeth and the length of non-ratchet parts are varied to study their effects on the flow behavior. 
All the experimental results, including flow rate time series and normalized histograms, are shown in Fig.~\ref{fig:bifurcation-ratchet}.
A general observation is that, in contrast to straight channel bifurcation systems, the flows exhibit a sharp peak in the histogram, indicating that the flow configurations are more deterministic and stable.
When the outlet channels have different numbers of ratchet, as the channel system shown in Fig.~\ref{fig:bifurcation-ratchet}(a), the flow robustly splits into different fractions in the two outlet channels.
Interestingly, the flow rate ratio in the two outlet channels is almost equal to the ratio between the number of ratchet teeth.

Does this unequal splitting of flow arise from the difference in the length of the channels?
To answer this question, we keep the channels lengths unchanged, but modify the number of ratchet teeth in channel B, so that channels B and C has the same number of ratchet teeth.
In this case, the flow robustly split into the two outlet channels with a 1:1 ratio, as shown in Fig.~\ref{fig:bifurcation-ratchet}(b).
This result suggests that the ratchet teeth in the outlet channels play a dominant role in determining the splitting ratio at bifurcations.

Does the length of the straight part of the outlet channels matter?
To answer this question, we further modify the channel system in Fig.\ref{fig:bifurcation-ratchet}(b) by make the length of the straight part of all the three channels identical, as shown in Fig.~\ref{fig:bifurcation-ratchet}(c).
The flow again splits with a 1:1 ratio, confirming the dominant role of ratchet teeth.


\subsection{Defect dynamics: wall-slip and ratchet pinning}

The dynamics of topological defects near the channel wall may provide insights into the steering mechanism of the ratchet teeth, as well as the difference between the flow in straight and ratchet channels.
In ratchet channels, +1/2 defects are observed to constantly form at the indentations on the channel walls, as shown in Fig.~\ref{fig:kymograph}(a). 
All these +1/2 defects are pinned for around 10 seconds, before turning and merging with the main channel flow. 
These pinned defects are prohibited from moving further, and the extensile active stress generated by the kinesin motors drives the microtubules outside the indentations to flow in a consistent direction.
\blue{See supplemental video ratchet flow (08\_A).}
This wall pinning phenomenon is similar an earlier observation in a circular well \cite{Hardouin2022}.
The difference is that in the present work, the indentation is made asymmetric, thus setting a preferred flow direction.
The kymograph in Fig.~\ref{fig:kymograph}(a) also verifies the constant formation and pinning of the +1/2 defects.
+1/2 defects also form in straight channels and move to the channel walls. 
However, they are not pinned on the walls, but can instead slip along the walls, as shown in the kymograph in Fig.~\ref{fig:kymograph}(b), by the dark lines alternating between up and down motions.
\blue{See supplemental video straight flow (09\_A).}
This difference in wall defect dynamics may explain why ratchet channels play a more dominant role in determining the flow behavior at bifurcations.

\begin{figure}[htb]
    \includegraphics[width=.45\textwidth]{kymograph}
    \caption{
    \textbf{Wall defect dynamics.}
    (a) A snapshot of the active nematic flow in a ratchet channel and the kymograph at the blue dashed line. 
    (b) A snapshot of the active nematic flow in a straight channel and the kymograph at the blue dashed line.
    }
    \label{fig:kymograph}
\end{figure}

\subsection{Frustrated flow states}

Flow frustrations are induced by coupling multiple channels with spontaneous flows, and are characteristic to active flow networks \cite{Jorge2024, Beppu2024}.
In the fully ratchet asymmetric channel system, we clearly observed a frustrated flow state in the shorter channel.
A close up view of the orientational ordering in the active nematics is shown in Fig.~\ref{fig:frustrated-flow-state}.


\begin{figure*}[htb]
    \includegraphics[width=.8\textwidth]{frustrated_flow_state}
    \caption{
    \textbf{Frustrated flow state.}
    Active nematic orentational ordering at the bifurcation. The channel system is the same as in Fig.~\ref{fig:bifurcation-asymmetric}(a), where the flow comes in from channel A at $\phi=\phi_0$ and splits into channels B and C with splitting ratio around 3:1. On the right, we show the close up views of the active nematics to highlight the deference in the orentational ordering in normal normal flow state (channels A and B) and frustrated flow state (channel C).
    }
    \label{fig:frustrated-flow-state}
\end{figure*}

% \begin{figure*}[!ht]
%     \includegraphics[width=\textwidth]{6-angle-ratchet-competition}
%     \caption{
%     \textbf{The role of turning angles.}
%     (a) A bifurcation with a 9-teeth ratchet inlet and 2 straight outlets of the same length. 
%     The outlets have different turning angles with respect to the inlet channel A: $\angle AOB=90^\circ$ and $\angle AOC=180^\circ$.
%     The flow rate histogram and time series suggest that the flow prefers the $180^\circ$ channel C, i.e. the channel parallel to the inlet channel A, rather than channel B which requires a $90^\circ$ turn. 
%     (b) Adding various numbers of ratchets to channel B to compete with the $90^\circ$ turning angle. 
%     From left to right, 1, 3, 5, 7 ratchet teeth are added to the end of channel B. 
%     Below the schematics of bifurcation channels are the $\phi_B$-$\phi_C$ flow rate histograms corresponding to the design above. 
%     As the number of ratchet teeth in channel B is increased, the splitting ratio between B and C is increase from 0 to $\infty$.
%     }
%     \label{fig:angle-ratchet-competition}
% \end{figure*}

\bibliography{ref}
\bibliographystyle{unsrtnat}

\end{document}
%
% ****** End of file apssamp.tex ******